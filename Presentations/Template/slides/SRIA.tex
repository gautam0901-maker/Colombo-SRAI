%%%%%%
%
% $Autor: Wings $
% $Datum: 2020-01-18 11:15:45Z $
% $Pfad: githubtemplate/Template/Presentations/Template/slides/rename.tex $
% $Version: 4620 $
%
%
% !TeX encoding = utf8
% !TeX root = Rename
%
%%%%%%



\Mysection{European long-term budgeting}

\STANDARD{EU institutions and bodies}
{ 
  
  \begin{figure}[H]
  	\centering
  	\includegraphics[width=10cm]{images/EuropeanBudgeting}
  	\caption{EU institutions and bodies \cite{EuropeanCommission:2025}}
  \end{figure}
  

  The EU budget combines resources at EU level and enables EU countries to achieve more together than they could if they acted alone, for instance financing infrastructure or research projects \cite{EuropeanCommission:2025}. 



	
	\begin{figure}[H]
		\centering
		\includegraphics[width=9cm]{images/EuropeanComission}
		\caption{European Comission makes proposals for EU investments \cite{EuropeanCommission:2025}}
	\end{figure}
	
	\begin{itemize}
		\item proposes the Multiannual Financial Framework (MFF)
		\item Resources Decision 
		\item as specific sectoral legislation sets rules for specification of EU investments \cite{EuropeanCommission:2025}
	\end{itemize}
	


	\begin{figure}[H]
		\centering
		\includegraphics[width=9cm]{images/EUDiscussion}
		\caption{Parliament and Council of the EU negotiate \cite{EuropeanCommission:2025}}
	\end{figure}
	
	\begin{itemize}
		\item Council of the EU discusses the proposals until all EU countries agree and adopts its positions
		\item In parallel, the European Parliament discusses the proposals and adopts its positions \cite{EuropeanCommission:2025}
	\end{itemize}
	
}
\STANDARD{Long-term budget}
{
	\begin{itemize}
		\item The \textbf{Multiannual Financial Framework (MFF)} defines the EU’s long-term spending priorities and limits
		\item It sets maximum annual expenditure for the entire EU budget and its main policy areas
		\item The current MFF covers the period \textbf{2021--2027}
		\item Together with NextGenerationEU, the long-term budget amounts to around \textbf{€2 trillion} \cite{EuropeanUnion:2025}
	\end{itemize}
	

}

\STANDARD{Funding programmes and open calls of the current period}
{
	\begin{columns}[c]
		
		% ----- LEFT -----
		\begin{column}{0.60\textwidth}
			\begin{itemize}
				\item \textbf{Horizon Europe} — Budget: \textbf{€95.5 billion.}  
				EU research and innovation framework programme (2021--2027) focusing on 
				climate action, UN SDGs, competitiveness, and knowledge creation 
				\cite{Com25a}
			\end{itemize}
		\end{column}
		
		% ----- RIGHT -----
		\begin{column}{0.38\textwidth}
			\centering
			\includegraphics[width=\linewidth]{images/horizon_europe_logo} \\[2mm]
			{\footnotesize
				\textbf{Figure:} Horizon Europe \cite{Europainfo_HorizonEurope:2020}
			}
		\end{column}
		
	\end{columns}
}

\STANDARD{Types of European Partnerships}
{
	\begin{itemize}
		
		\item \textbf{Institutionalised Partnerships}
		\begin{itemize}
			\item Long-term structured collaborations
			\item Strong integration between partners
			\item Implemented through dedicated EU bodies
			\item Includes EIT Knowledge and Innovation Communities (KICs)
		\end{itemize}
		
		\item \textbf{Co-funded Partnerships}
		\begin{itemize}
			\item Joint research and innovation programmes
			\item Funded by EU + national authorities
			\item Typically run joint international calls
		\end{itemize}
		
		\item \textbf{Co-programmed Partnerships}
		\begin{itemize}
			\item Joint planning of research and innovation priorities
			\item Based on cooperation with industry associations
			\item EU provides funding through Horizon Europe calls \cite{EuropeanCommissionPartnerships:2025}
		\end{itemize}
		
	\end{itemize}
}

\STANDARD{Horizon Europe Partnerships}
{
	\begin{itemize}
		
		\item The Horizon Europe Strategic Plan 2025--2027 introduces new candidate co-funded and co-programmed partnerships.
		\item The full portfolio will consist of \textbf{60 European Partnerships}.
		
		\item \textbf{Partnership areas:}
		\begin{itemize}
			\item Climate, energy and mobility
			\item Culture, creativity and inclusive society
			\item Digital, industry and space
			\item Food, bioeconomy, natural resources, agriculture and environment
			\item Health \cite{EuropeanCommissionPartnerships:2025}
		\end{itemize}
		
	\end{itemize}
}

\STANDARD{What is a SRIA?}
{
	\begin{itemize}
		
		\item A \textbf{Strategic Research and Innovation Agenda (SRIA)} is the core strategy document of a European Partnership
		\item It defines objectives, impact areas, expected outcomes, activities, outputs, and milestones
		\item Provides a \textbf{long-term systemic plan} with clear principles, impact logic, and pathways
		\item Forms the basis for annual or multi-annual work programmes
	\end{itemize}
}


\STANDARD{Principles for SRIA creation}
{
	\begin{itemize}
		\item Strategic and impact-driven
		\item Transparent and inclusive
		\item Evidence-based and foresight-oriented
		\item Flexible and adaptable to change
		\item Shared ownership and commitment
		\item Strong stakeholder engagement
	\end{itemize}
}


\STANDARD{General structure of a SRIA}
{
	\begin{itemize}
		\item Vision and strategic objectives
		\item Impact areas and expected outcomes
		\item Portfolio of activities and workstreams
		\item Milestones and timeline
		\item Governance structure and responsibilities
		\item Monitoring, evaluation and adaptation mechanisms
	\end{itemize}
}


\STANDARD{Process for developing a SRIA}
{
	\begin{itemize}
		\item Draft preparation by partnership members
		\item Foresight activities and evidence gathering
		\item Stakeholder consultation and co-creation
		\item Alignment with partnership vision and EU priorities
		\item Finalisation, approval and publication
		\item Implementation via annual or multi-annual work plans
	\end{itemize}
}


\STANDARD{Methods useful for SRIA development}
{
	\begin{itemize}
		\item Foresight (scenarios, horizon scanning)
		\item Surveys, interviews and public consultations
		\item Expert panels and stakeholder workshops
		\item Trend analysis and literature review
		\item Portfolio mapping and gap analysis
	\end{itemize}
}



