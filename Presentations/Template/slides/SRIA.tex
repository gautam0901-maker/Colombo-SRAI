%%%%%%
%
% $Autor: Wings $
% $Datum: 2020-01-18 11:15:45Z $
% $Pfad: githubtemplate/Template/Presentations/Template/slides/rename.tex $
% $Version: 4620 $
%
%
% !TeX encoding = utf8
% !TeX root = Rename
%
%%%%%%



\Mysection{European long-term budgeting}



\Mysection{\textbf{Why} the EU funds projects}

%---------------------------- EU + MISSIONS ------------------------------
\begin{frame}{EU Research and Missions}
    \framesubtitle{Gautam}
    \begin{columns}
        \begin{column}{0.45\textwidth}
            \centering
            \includegraphics[width=\textwidth]{images/eu2.jpg}\\[4pt]
            \small The \textbf{European Union} invests money in research and innovation.
        \end{column}
        \begin{column}{0.55\textwidth}
            \centering
            \includegraphics[width=\textwidth]{images/eu1.png}\\[4pt]
            \small EU \textbf{missions} focus on big problems like climate, cancer,
            oceans, cities and healthy soil.
        \end{column}
    \end{columns}
\end{frame}




\STANDARD{EU institutions and bodies}
{ \framesubtitle{Malena}
  
  \begin{figure}[H]
  	\centering
  	\includegraphics[width=10cm]{images/EuropeanBudgeting}
  	\caption{EU institutions and bodies \cite{EuropeanCommission:2025}}
  \end{figure}
  

  The EU budget combines resources at EU level and \textbf{enables EU countries to achieve more together than they could if they acted alone}, for instance financing infrastructure or research projects \cite{EuropeanCommission:2025}. 



	
	\begin{figure}[H]
		\centering
		\includegraphics[width=9cm]{images/EuropeanComission}
		\caption{European Comission makes proposals for EU investments \cite{EuropeanCommission:2025}}
	\end{figure}
	
	\begin{itemize}
		\item proposes the Multiannual Financial Framework (MFF)
		\item Resources Decision 
		\item sets rules for specification of EU investments \cite{EuropeanCommission:2025}
	\end{itemize}
	


	\begin{figure}[H]
		\centering
		\includegraphics[width=9cm]{images/EUDiscussion}
		\caption{Council and Parliament of the EU negotiate \cite{EuropeanCommission:2025}}
	\end{figure}
	
	\begin{itemize}
		\item Council of the EU discusses the proposals until all EU countries agree and adopts its positions
		\item In parallel, the European Parliament discusses the proposals and adopts its positions \cite{EuropeanCommission:2025}
	\end{itemize}
	
}

\STANDARD{Long-term budget}
{\framesubtitle{Malena}
    \begin{itemize}
        \item The \textbf{Multiannual Financial Framework (MFF)} defines the EU’s long-term spending priorities and limits
        \item The current MFF covers the period \textbf{2021--2027}
        \item Together with NextGenerationEU, the long-term budget amounts to around \textbf{€2 trillion} \cite{EuropeanCommissionPartnerships:2025}
    \end{itemize}
    
    
}


\STANDARD{Horizon Europe}
{\framesubtitle{Malena}
	\begin{columns}[c]
		
		% ----- LEFT -----
		\begin{column}{0.60\textwidth}
			\begin{itemize}
				\item \textbf{Horizon Europe} — Budget: \textbf{€95.5 billion.}  
				EU research and innovation strategic plan (2021--2027) focusing on 
				climate action, United Nations Sustainable Development Goals (UN SDGs), competitiveness, and knowledge creation 
				\cite{Europainfo_HorizonEurope:2020}
			\end{itemize}
		\end{column}
		
		% ----- RIGHT -----
		\begin{column}{0.38\textwidth}
			\centering
			\includegraphics[width=\linewidth]{images/horizon_europe_logo} \\[2mm]
			{\footnotesize
				\textbf{Figure:} Horizon Europe \cite{Europainfo_HorizonEurope:2020}
			}
		\end{column}
		
	\end{columns}
}


%\STANDARD{Types of European Partnerships}
%{\framesubtitle{Malena}
%	\begin{itemize}
%		
%		\item \textbf{Institutionalised Partnerships}
%		\begin{itemize}
%			\item Long-term structured collaborations
%			\item Strong integration between partners
%			\item Implemented through dedicated EU bodies
%			\item Includes EIT Knowledge and Innovation Communities (KICs)
%		\end{itemize}
%		
%		\item \textbf{Co-funded Partnerships}
%		\begin{itemize}
%			\item Joint research and innovation programmes
%			\item Funded by EU + national authorities
%			\item Typically run joint international calls
%		\end{itemize}
%		
%		\item \textbf{Co-programmed Partnerships}
%		\begin{itemize}
%			\item Joint planning of research and innovation priorities
%			\item Based on cooperation with industry associations
%			\item EU provides funding through Horizon Europe calls \cite{EuropeanCommissionPartnerships:2025}
%		\end{itemize}
%		
%	\end{itemize}
%}

\STANDARD{Horizon Europe Partnerships}
{\framesubtitle{Malena}
	\begin{itemize}
		
		\item The full portfolio consist of \textbf{60 European Partnerships}.
		
		\item \textbf{Partnership areas:}
		\begin{itemize}
			\item Climate, energy and mobility
			\item Culture, creativity and inclusive society
			\item Digital, industry and space
			\item Food, bioeconomy, natural resources, agriculture and environment
			\item Health \cite{EuropeanCommissionPartnerships:2025}
		\end{itemize}
		
	\end{itemize}
}

	%-------------------------- FROM CALL TO PROJECT -------------------------
\begin{frame}{From Call to Project}
    \framesubtitle{Gautam}
    \begin{itemize}
        \item The EU publishes a \textbf{call text} with a topic and rules.
        \item A group of partners writes a \textbf{project proposal}.
        \item They send the proposal on time through the online portal.
        \item After the deadline, the proposal enters the \textbf{evaluation process}.
        \item Only the best proposals become \textbf{funded projects}.
    \end{itemize}
\end{frame}




\Mysection{\textbf{Who} checks the proposals}

%--------------------------- WHO EVALUATES -------------------------------
\begin{frame}{Who Evaluates the Proposal?}
    \framesubtitle{Gautam}
    \begin{itemize}
        \item Proposals are checked by \textbf{independent experts}.
        \item Experts come from \textbf{different countries} and backgrounds.
        \item They are chosen for their \textbf{knowledge} in the topic.
        \item They must follow \textbf{conflict of interest} rules
        (they cannot judge their own project).
        \item Each proposal is read by \textbf{several experts}, not only one person.
    \end{itemize}
\end{frame}

%--------------------------- PIPELINE (TIKZ) -----------------------------
\begin{frame}{Evaluation Pipeline}
    \framesubtitle{Gautam}
    \centering
    \begin{tikzpicture}[node distance=1.0cm]
        \node[flowbox] (call)      {Call for proposals};
        \node[flowbox, below of=call] (submit)   {Proposal submitted};
        \node[flowbox, below of=submit] (check)  {Basic checks};
        \node[flowbox, below of=check] (review)  {Experts review};
        \node[flowbox, below of=review] (panel)  {Panel ranks proposals};
        \node[flowbox, below of=panel] (grant)   {Best ones get grant};
        
        \draw[->, thick] (call)   -- (submit);
        \draw[->, thick] (submit) -- (check);
        \draw[->, thick] (check)  -- (review);
        \draw[->, thick] (review) -- (panel);
        \draw[->, thick] (panel)  -- (grant);
    \end{tikzpicture}
    
    \vspace{4pt}
    \small Each step removes proposals that do not fit the rules or have low quality.
\end{frame}


\STANDARD{Class Evaluation}
{ \framesubtitle{Gautam}
    
    
}

\Mysection{\textbf{How} the evaluation works}
%------------------------------ SCORING ----------------------------------
\begin{frame}{Scoring System}
    \framesubtitle{Gautam}
    \begin{itemize}
        \item Experts give scores from \textbf{0} (very poor) to \textbf{5} (excellent).
        \item Scores are given for each \textbf{criterion}.
        \item A proposal must reach a \textbf{minimum score} to stay in the game.
        \item Later, experts meet and agree on a \textbf{common score}.
        \item Then the EU creates a \textbf{ranked list} from the best to the worst.
    \end{itemize}
\end{frame}


\Mysection{The three main criteria: \textbf{Excellence}, \textbf{Impact}, \textbf{Implementation}}
%----------------------- THREE CRITERIA (TIKZ) ---------------------------
\begin{frame}{Three Main Evaluation Criteria}
    \framesubtitle{Gautam}
    \centering
    \begin{tikzpicture}[node distance=2.4cm]
        \node[crit] (ex)  {Excellence};
        \node[crit, right of=ex] (im) {Impact};
        \node[crit, below of=ex, xshift=1.2cm] (impl) {Implementation};
        
        \draw[thick] (ex) -- (im);
        \draw[thick] (im) -- (impl);
        \draw[thick] (impl) -- (ex);
    \end{tikzpicture}
    
    \vspace{4pt}
    \small Every proposal is judged on these three \textbf{pillars}.
\end{frame}

%-------------------------- EXCELLENCE (IDEA) ----------------------------
\begin{frame}{Excellence: The Idea}
    \framesubtitle{Gautam}
    \begin{itemize}
        \item The proposal must start with a \textbf{simple and clear problem}.
        \item It must explain what people \textbf{already know} about this problem.
        \item It must show what is \textbf{missing} or what is not solved yet.
        \item It should have \textbf{clear goals} that the team wants to reach.
        \item The idea should follow a \textbf{logical plan} and make sense.
    \end{itemize}
\end{frame}

%------------------------ EXCELLENCE (METHOD) ----------------------------
\begin{frame}{Excellence: The Method}
    \framesubtitle{Gautam}
    \begin{itemize}
        \item The proposal must show \textbf{how} the team will work.
        \item It should list the \textbf{tools} and \textbf{techniques} they will use.
        \item It must explain \textbf{why} these tools are the right choice.
        \item It should say how they will \textbf{collect} and \textbf{study} data.
        \item It must mention possible \textbf{problems} and how they will solve them.
    \end{itemize}
\end{frame}

%------------------------------- IMPACT ----------------------------------
\begin{frame}{Impact: The Change}
    \framesubtitle{Gautam}
    \begin{itemize}
        \item Describes what will \textbf{change} if the project is successful.
        \item Links the project to \textbf{EU missions} or strategies.
        \item Explains who will \textbf{benefit} and how.
        \item Gives simple \textbf{numbers} to measure success
        (for example fewer accidents, less energy use).
    \end{itemize}
\end{frame}

%---------------------------- IMPACT PLANS -------------------------------
\begin{frame}{Impact Plans: Use and Communication}
    \framesubtitle{Gautam}
    \begin{itemize}
        \item Shows how results will be \textbf{used} after the project
        (for example product, service, policy).
        \item Explains how results will be \textbf{shared}:
        reports, websites, open events, training.
        \item Talks about the \textbf{target groups}:
        cities, companies, citizens, students.
        \item Makes sure the project does not end in a \textbf{drawer}.
    \end{itemize}
\end{frame}

	%---------------------------- IMPLEMENTATION -----------------------------
\begin{frame}{Implementation: Work Plan}
    \framesubtitle{Gautam}
    \begin{itemize}
        \item Breaks the project into \textbf{work packages}.
        \item Each work package has tasks, a leader and a time period.
        \item Shows a simple \textbf{timeline} (Gantt style).
        \item Explains how partners will \textbf{manage} the project:
        meetings, reports, risk checks.
    \end{itemize}
\end{frame}

%---------------------------- IMPLEMENTATION 2 ---------------------------
\begin{frame}{Implementation: The Team}
    \framesubtitle{Gautam}
    \begin{itemize}
        \item A good consortium mixes:
        \begin{itemize}
            \item \textbf{Universities} (knowledge)
            \item \textbf{Companies} (market and products)
            \item \textbf{Cities or users} (real-life testing)
        \end{itemize}
        \item Each partner has a \textbf{clear role}.
        \item The coordinator has experience in \textbf{project management}.
    \end{itemize}
\end{frame}

%------------------------------- TABLE -----------------------------------
\begin{frame}{Short Overview of the Three Criteria}
    \framesubtitle{Gautam}
    \centering
    \begin{tabular}{|p{3cm}|p{7cm}|}
        \hline
        \textbf{Excellence} &
        Quality of the idea and method. Clear problem and clear objectives. \\ \hline
        \textbf{Impact} &
        Useful change in society or the market. Strong link to EU goals. \\ \hline
        \textbf{Implementation} &
        Realistic work plan, good team and management. \\ \hline
    \end{tabular}
\end{frame}



\Mysection{SRIA}




\STANDARD{Example for an SRIA}
{\framesubtitle{Malena}
 
}




%GAUTAM 2nd presentation





	
	%-------------------------- SMART TRAFFIC IMAGE -------------------------
	\begin{frame}{Example Project: Smart Traffic}
        \framesubtitle{Gautam}
		\begin{columns}
			\begin{column}{0.45\textwidth}
				\includegraphics[width=\textwidth]{images/eu3.jpg}
			\end{column}
			\begin{column}{0.55\textwidth}
				\small
				\begin{itemize}
					\item Goal: make a city \textbf{safer} at busy crossings.
					\item University builds an \textbf{AI model} to predict risky situations.
					\item Company builds \textbf{smart cameras} and software.
					\item City installs the system at real junctions.
				\end{itemize}
			\end{column}
		\end{columns}
	\end{frame}


\Mysection{ A simple \textbf{smart traffic} example}
	%-------------------------- SMART TRAFFIC + CRIT ------------------------
	\begin{frame}{Smart Traffic Project and the Criteria}
        \framesubtitle{Gautam}
		\small
		\begin{itemize}
			\item \textbf{Excellence}: clear accident problem, new AI-based solution.
			\item \textbf{Impact}: fewer injuries, supports green and smart city mission.
			\item \textbf{Implementation}: clear work packages,
			3-year plan, partners with real responsibilities.
		\end{itemize}
	\end{frame}
	
	%--------------------------- COMMON MISTAKES -----------------------------
	\begin{frame}{Typical Weak Points in Proposals}
        \framesubtitle{Gautam}
		\begin{itemize}
			\item Problem is \textbf{not clear} or too broad.
			\item Many nice words but \textbf{no concrete impact}.
			\item Work plan looks \textbf{too optimistic} for the time and budget.
			\item Missing important partners (for example no city partner for a city project).
			\item Risks are ignored or only written in one short line.
		\end{itemize}
	\end{frame}
	
	%------------------------- WHAT MAKES IT STRONG --------------------------
	\begin{frame}{What Makes a Proposal Strong}
        \framesubtitle{Gautam}
		\begin{itemize}
			\item Simple and \textbf{clear story} that everyone understands.
			\item Real \textbf{innovation}, but still realistic.
			\item Impact that matches \textbf{EU missions}.
			\item Work plan that looks \textbf{doable}.
			\item Team that has the \textbf{right skills}.
		\end{itemize}
	\end{frame}
	
	%----------------------------- STUDENT IMAGE -----------------------------
	\begin{frame}{Students in EU Projects}
        \framesubtitle{Gautam}
		\begin{columns}
			\begin{column}{0.5\textwidth}
				\includegraphics[width=\textwidth]{images/eu4.jpg}
			\end{column}
			\begin{column}{0.5\textwidth}
				\small
				\begin{itemize}
					\item Students cannot be \textbf{main applicants}.
					\item But they can work inside a \textbf{university team}.
					\item They help with data, coding, tests and reports.
					\item This gives strong \textbf{experience} and a good CV.
				\end{itemize}
			\end{column}
		\end{columns}
	\end{frame}

\Mysection{\textbf{How students} can join such projects}
	%--------------------------- STUDENT PATH TIKZ --------------------------
	\begin{frame}{How a Student Can Join a Project}
        \framesubtitle{Gautam}
		\centering
		\begin{tikzpicture}[node distance=1.5cm]
			\node[studentflow] (interest) {1. Interest in a topic};
			\node[studentflow, below of=interest] (prof) {2. Talk to a professor};
			\node[studentflow, below of=prof] (group) {3. Join a research group};
			\node[studentflow, below of=group] (project) {4. Work on an EU project};
			
			\draw[->, thick] (interest) -- (prof);
			\draw[->, thick] (prof) -- (group);
			\draw[->, thick] (group) -- (project);
		\end{tikzpicture}
	\end{frame}
	
	%----------------------------- STUDENT ROLES -----------------------------
	\begin{frame}{Typical Student Roles}
        \framesubtitle{Gautam}
		\begin{itemize}
			\item \textbf{Data work}: collecting, cleaning and simple analysis.
			\item \textbf{Programming}: small modules, scripts, dashboards.
			\item \textbf{Testing}: trying out prototypes in the lab or in the field.
			\item \textbf{Support}: helping with figures, slides and documentation.
			\item \textbf{Thesis}: writing a project-related bachelor or master thesis.
		\end{itemize}
	\end{frame}
	
	%------------------------------- BENEFITS --------------------------------
	\begin{frame}{Benefits for Students}
        \framesubtitle{Gautam}
		\begin{itemize}
			\item Learn how real \textbf{EU projects} work.
			\item Build an \textbf{international network}.
			\item Gain experience that looks strong on a \textbf{CV}.
			\item Understand how research ideas become \textbf{funded projects}.
		\end{itemize}
	\end{frame}
	
	%------------------------------- SUMMARY ---------------------------------
%	\begin{frame}{Summary}
%        \framesubtitle{Gautam}
%		\begin{itemize}
%			\item EU projects are evaluated on \textbf{Excellence}, \textbf{Impact} and
%			\textbf{Implementation}.
%			\item Only clear, realistic and high-impact proposals are funded.
%			\item Students cannot lead these proposals but can \textbf{join} them.
%			\item Working in such a project is a strong step for a \textbf{future career}.
%		\end{itemize}
%	\end{frame}
%	
	%--------------------------------- END -----------------------------------
%	\begin{frame}
%		\centering
%		%\Large Thank you for listening.\\[8pt]
%		\large Any Questions?
%	\end{frame}
	
